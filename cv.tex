%%%%%%%%%%%%%%%%%
% This is an sample CV template created using altacv.cls
% (v1.7.4, 30 July 2025) written by LianTze Lim (liantze@gmail.com). Compiles with pdfLaTeX, XeLaTeX and LuaLaTeX.
%
%% It may be distributed and/or modified under the
%% conditions of the LaTeX Project Public License, either version 1.3
%% of this license or (at your option) any later version.
%% The latest version of this license is in
%%    http://www.latex-project.org/lppl.txt
%% and version 1.3 or later is part of all distributions of LaTeX
%% version 2003/12/01 or later.
%%%%%%%%%%%%%%%%

%% Use the "normalphoto" option if you want a normal photo instead of cropped to a circle
% \documentclass[10pt,a4paper,withhyper,normalphoto]{altacv}

\documentclass[10pt,a4paper,withhyper]{altacv}
%% AltaCV uses the fontawesome5 and simpleicons packages.
%% See http://texdoc.net/pkg/fontawesome5 and http://texdoc.net/pkg/simpleicons for full list of symbols.

\geometry{left=1.25cm,right=1.25cm,top=1.5cm,bottom=1.5cm,columnsep=1.2cm}

\usepackage{paracol}
\usepackage{fancyhdr}
\usepackage[rm]{roboto}
\usepackage[defaultsans]{lato}
% \usepackage{sourcesanspro}
\renewcommand{\familydefault}{\sfdefault}

% Change the colours if you want to
\definecolor{SlateGrey}{HTML}{2E2E2E}
\definecolor{LightGrey}{HTML}{666666}
\definecolor{DarkPastelRed}{HTML}{450808}
\definecolor{PastelRed}{HTML}{8F0D0D}
\definecolor{GoldenEarth}{HTML}{E7D192}
\colorlet{name}{black}
\colorlet{tagline}{PastelRed}
\colorlet{heading}{DarkPastelRed}
\colorlet{headingrule}{GoldenEarth}
\colorlet{subheading}{PastelRed}
\colorlet{accent}{PastelRed}
\colorlet{emphasis}{SlateGrey}
\colorlet{body}{LightGrey}

% Change some fonts, if necessary
\renewcommand{\namefont}{\Huge\rmfamily\bfseries}
\renewcommand{\personalinfofont}{\footnotesize}
\renewcommand{\cvsectionfont}{\LARGE\rmfamily\bfseries}
\renewcommand{\cvsubsectionfont}{\large\bfseries}

% Change the bullets for itemize and rating marker
% for \cvskill if you want to
\renewcommand{\cvItemMarker}{{\small\textbullet}}
\renewcommand{\cvRatingMarker}{\faCircle}
% ...and the markers for the date/location for \cvevent
% \renewcommand{\cvDateMarker}{\faCalendar*[regular]}
% \renewcommand{\cvLocationMarker}{\faMapMarker*}


%% Use (and optionally edit if necessary) this .tex if you
%% want an originally numerical reference style like IEEE
%% for your publication list
\input{pubs-num}

%% cv.bib contains your publications
\addbibresource{cv.bib}


\begin{document}

\pagestyle{fancy}
\fancyfoot{}
\fancyfoot[LE,RO]{\textbf{\thepage}}
\fancyfoot[LO,RE]{Pascal-Emmanuel Lachance}

\name{Pascal-Emmanuel Lachance}
\tagline{Maîtrise en Génie Électrique - Cheminement interdisciplinaire en imagerie médicale}

%% You can add multiple photos on the left or right
%\photoR{2.8cm}{Globe_High}
% \photoL{2.5cm}{Yacht_High,Suitcase_High}

\personalinfo{%
  % Not all of these are required!
  \email{pascalemmanuel.lachance@gmail.com}
  \phone{+1-581-748-9134}
  \mailaddress{2575, Rue Galt O, J1K 1L7}
  \location{Sherbrooke, CANADA}
  \homepage{www.raesangur.com}
  % \twitter{@twitterhandle}
  \linkedin{https://www.linkedin.com/in/pascal-emmanuel-l-b9686aa4/}
  \github{https://github.com/Raesangur}
  %\orcid{0000-0000-0000-0000}
  %% You can add your own arbitrary detail with
  %% \printinfo{symbol}{detail}[optional hyperlink prefix]
  % \printinfo{\faPaw}{Hey ho!}[https://example.com/]

  %% Or you can declare your own field with
  %% \NewInfoFiled{fieldname}{symbol}[optional hyperlink prefix] and use it:
  %\NewInfoField{gitlab}{\faGitlab}[https://gitlab.com/]
  %\gitlab{your_id}
  %%
}

\makecvheader

\columnratio{0.575}
\begin{paracol}{2}

%\begin{itemize}
%    \item Concevoir des produits électroniques miniature de haute fiabilité (PCB multicouches, PCB flexibles, FPGA)
%    \item Programmer et corriger des logiciels
%    \item Analyser, diagnostiquer et résoudre des problèmes électriques et informatiques
%\end{itemize}

\cvsection{Expérience Professionnelle}

\cvevent{Tutorat en conception de PCB}{Université de Sherbrookke}{Janvier 2025 -- Mai 2025}{Sherbrooke}
\begin{itemize}
\item Suivi personalisé avec étudiants en difficulté d'apprentissage pour les cours de spécialisation de dernière année
\end{itemize}

\divider
\cvevent{Mentor en fabrication}{Studio de Création}{Hivers 2024 \& 2025}{Sherbrooke}
\begin{itemize}
\item Maintenance des machines du FabLab de l'Université
\item Former des étudiants à utiliser l'équipement de fabrication pour fabriquer des prototypes
\item Assister des étudiants à conçevoir leurs projets
\end{itemize}

\divider
\cvevent{Caractérisation de ASIC en photonique sur silicium}{Groupe de recherche en Appareillage Médical, 3iT}{Stage Mai 2024 -- Août 2024}{Sherbrooke}
\begin{itemize}
\item Validation de systèmes intégrés
\item Programmation SystemVerilog
\end{itemize}

\divider
\cvevent{Programmation de BMS}{Calogy Solutions}{Stage Août 2023 -- Décembre 2023}{Sherbrooke}
\begin{itemize}
\item Programmation en C de systèmes de gestion de batteries pour le domaine automobile et aéronautique
\end{itemize}

\divider
\cvevent{Conception électronique miniaturisée}{Krag Systems Inc.}{Stage Août 2022 -- Décembre 2022}{Sherbrooke}
\begin{itemize}
\item Conception de PCB miniatures multicouches devant respecter des normes EMC
\item Conception de PCB flexibles pour intégration électromécanique
\end{itemize}

\divider
\cvevent{Conception électronique de systèmes spaciaux}{ABB}{Stage Janvier 2022 -- Mai 2022}{Québec}
\begin{itemize}
\item Conception d'un PCB multicouches destiné à un satellite
\item Conception, révision et tests de dispositifs électroniques industriels
\end{itemize}

\newpage
\cvevent{Développement d'un magnétomètre quantique spatial}{Université de Sherbrooke}{Stage Mai 2021 -- Août 2021}{Sherbrooke}
\begin{itemize}
\item Conception, révision et tests d'un PCB 12 couches pour prototype de magnétomètre quantique destiné à un satellite
\item Dans le cadre du projet QMSat CubeSat, en association avec l'ASC
\end{itemize}

\divider
\cvevent{Conception de systèmes de suivis dans les hopitaux}{Humanitas}{Mars 2020 -- Avril 2020}{Québec}
\begin{itemize}
\item Bénévolat au début de la pandémie de COVID-19
\item Conception de PCBs faisants des nœuds de réseau dans les hôpitaux du Canada
\end{itemize}

\divider
\cvevent{Développement de bancs de tests automatisés}{Conception Électronique Privé}{Janvier 2019 -- Août 2020}{Québec}
\begin{itemize}
\item Programmation en C de systèmes de tests automatisés
\item Conception de systèmes de sécurités pour véhicules lourds
\end{itemize}

\divider
\cvevent{Tutorat en AEC en Programmation}{Cégep Limoilou}{Juin 2019 -- Mars 2020}{Québec}
\begin{itemize}
\item Expliquer des concepts de programmation en C\#
\item Accompagner des élèves dans leurs travaux
\end{itemize}

\divider
\cvevent{Animateur de camp de jour (\,Thématique: robotique\,)}{Safari Éducation \& Loisirs Saint-Sacrement}{Étés 2017 et 2018}{Québec}
\begin{itemize}
\item Animer des activités aux thèmes de la robotique à des jeunes de 7 à 12 ans
\end{itemize}


%% Specify your last name(s) and first name(s) as given in the .bib to automatically bold your own name in the publications list.
%% One caveat: You need to write \bibnamedelima where there's a space in your name for this to work properly; or write \bibnamedelimi if you use initials in the .bib
%% You can specify multiple names, especially if you have changed your name or if you need to highlight multiple authors.
\mynames{Pascal-Emmanuel\bibnamedelima Lachance,
  Pascal-Emmanuel,
  P.-E.\bibnamedelimi L.,
  P-E\bibnamedelimi L}
%% MAKE SURE THERE IS NO SPACE AFTER THE FINAL NAME IN YOUR \mynames LIST

\nocite{*}

\printbibliography[heading=pubtype,title={\printinfo{\faBook}{Books}},type=book]

%% Switch to the right column. This will now automatically move to the second
%% page if the content is too long.
\switchcolumn

\cvsection{Langages}

\cvskill{Français}{5}
\cvskill{Anglais}{5}
\cvskill{Japonais}{1.5}

\divider
\cvskill{C}{5} 
\cvskill{C++}{5} 
\cvskill{Python}{4.5} 
\cvskill{C\#}{4} 
\cvskill{Bash}{3} 
\cvskill{ASM (x86, ARM, MIPS)}{3} 
\cvskill{VHDL \& SystemVerilog}{3} 
\cvskill{Java}{2.5} 
\cvskill{LaTeX}{4} 

\medskip
\cvsection{Éducation}

\cvevent{Maîtrise en Génie Électrique}{Université de Sherbrooke}{Mai 2025 -- Présent}{}
Développement d’un nouveau système d’acquisition modulaire et extensible pour l’interfaçage de matrices de photodétecteurs à base de diode à avalanche monophotonique

\divider

\vspace{18pt}
\hfill\begin{minipage}{2cm}
    \textbf{\color{heading}\small Cote Z: 4.12}
\end{minipage}
\vspace{-28pt}

\cvevent{Baccalauréat en Génie Informatique}{Université de Sherbrooke}{Août 2020 -- Décembre 2024}{}

\begin{itemize}
\item Spécialisation en conception de PCB
\item Spécialisation en conception d'ASIC
\item Spécialisation en vérification fonctionnelle
\item Projet \href{https://github.com/orgs/Reali-Plus/repositories}{Réali+}\,: Système d'acquisition de mouvement pour réalité virtuelle
\end{itemize}

\divider

\vspace{46pt}
\hfill\begin{minipage}{2cm}
    \textbf{\color{heading}\small Cote R: 31.5}
\end{minipage}
\vspace{-56pt}

\cvevent{Technique en Systèmes Ordinés - Électronique Programmable et Robotique}{Cégep Limoilou, Québec}{Août 2016 -- Juin 2019}{}
\begin{itemize}
    \item Responsable de programme à l'AGEECL (Association Étudiante)
    \item Projet \href{https://github.com/graphbit/GraphBit-V1}{GraphBit}\,: Développement d'un contrôleur d'écran sur FPGA
\end{itemize}

\newpage
\cvevent{Diplôme d'études secondaires}{Collège Saint-Charles Garnier}{2011 -- 2016}{Québec}
\begin{itemize}
    \item Fondateur du \textit{Club de robotique}, menant à la création du \href{https://collegegarnier.qc.ca/programmes/11-profils/robotique/}{profil Robotique et programmation}
\end{itemize}

\cvsection{Logiciels}

\cvtag{Suite Office}
\cvtag{Overleaf}
\cvtag{Microsoft Teams}
\cvtag{Jira}
\cvtag{Linux}
\cvtag{Windows}
\cvtag{FreeRTOS}
\cvtag{Proxmox}
\cvtag{FreeBSD}
\cvtag{Android}
\cvtag{Altium Designer}
\cvtag{KiCAD}
\cvtag{OrCAD}
\cvtag{LTSpice}
\cvtag{TINA-TI}
\cvtag{Git}
\cvtag{SVN}
\cvtag{Github Actions}
\cvtag{SolidWorks}
\cvtag{FreeCAD}
\cvtag{AutoCAD}
\cvtag{Visual Studio}
\cvtag{Vivado}
\cvtag{STM32CubeIDE}
\cvtag{Eclipse}
\cvtag{CMake}
\cvtag{gcc}
\cvtag{LLVM}


\end{paracol}


\cvsection{Projets \& Implication}

\hfill\begin{minipage}{2.5cm}
    \includegraphics[width=\linewidth]{figures/vamudes.pdf}
\end{minipage}
\vspace{-1cm}

\cvevent{VAMUdeS}{Groupe technique universitaire: Conception de drones}{2020 -- 2024}{}
\begin{itemize}
\item Pilote de la charge utile pour la compétition USC 2021
\item Opérateur station terrestre pour les compétitions AÉAC 2022 \& 2023
\item Responsable santé et sécurité 2021 -- 2024
\item Responsable informatique 2022 -- 2024
\item Responsable électrique 2023 -- 2024
\end{itemize}

\divider

\hfill\begin{minipage}{2.5cm}
    \includegraphics[width=\linewidth]{figures/sirius}
\end{minipage}
\vspace{-0.75cm}

\cvevent{SIRIUS}{Groupe technique universitaire: Conception de satellites}{2021 -- 2024}{}
\begin{itemize}
\item \textbf{Satellité QMSat lancé dans l'espace en 2024!}
\item En collaboration avec l'Agence Spatiale Canadienne -- Initiative du \textit{Canadian CubeSat Project}
\item Responsable de la charge utile 2021 -- 2023
\item Responsable santé et sécurité 2021 -- 2024
\item Responsable électrique 2022 -- 2024
\item Responsable informatique 2022
\end{itemize}

\divider

\hfill\begin{minipage}{1.25cm}
    \includegraphics[width=\linewidth]{figures/jdis}
\end{minipage}
\vspace{-1.25cm}

\cvevent{JDIS}{Groupe technique universitaire: Défis de programmation et hacking}{2022 -- 2024}{}
\begin{itemize}
\item Participation au NordSec CTF 2022, 2024 \& 2025
\item Participation au Hackfest CTF 2021 -- 2024
\item Participation au CSGames 2023 -- 2024
\item Participation au UnitedCTF 2022 -- 2023
\item Participation au JDISGames 2023 -- 2024 ($2^e$ place)
\end{itemize}

\divider

\hfill\begin{minipage}{1.25cm}
    \includegraphics[width=\linewidth]{figures/reali-plus}
\end{minipage}
\vspace{-1.1cm}

\cvevent{Réali+}{Projet Majeur de Conception}{2023 -- 2024}{}
\begin{itemize}
\item Système d'acquisition de mouvement et de rétroaction haptique pour réalité virtuelle et réalité augmentée
\item Proposeur du projet \& Chef d'équipe
\item Responsable de l'équipe électrique
\end{itemize}

\medskip

%\cvsection{Publications}

\end{document}
